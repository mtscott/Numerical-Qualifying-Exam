\documentclass[a4paper,12pt]{article} % This defines the style of your paper



\usepackage[top = 2.5cm, bottom = 2.5cm, left = 2.5cm, right = 2.5cm]{geometry} 



% The following two packages - multirow and booktabs - are needed to create nice looking tables.

\usepackage{multirow} % Multirow is for tables with multiple rows within one cell.

\usepackage{booktabs} % For even nicer tables.

\usepackage{setspace}

\usepackage{url}

\setlength{\parindent}{0in}

\usepackage{float}

\usepackage{amsmath,amsthm,amssymb,amsfonts, enumitem, fancyhdr, color, comment, graphicx, environ, textcomp, mathrsfs, physics}

\newcommand{\bbR}{\mathbb{R}}
\newcommand{\bbC}{\mathbb{C}}
\newcommand{\bbN}{\mathbb{N}}
\newcommand{\bbZ}{\mathbb{Z}}
\newcommand{\bbQ}{\mathbb{Q}}





\pagestyle{fancy}



\fancyhf{} % This makes sure we do not have other information in our header or footer.



\lhead{\footnotesize Qualifying Exam: Numerical Analysis } % Class: Assignment e.g. Abstract Algebra: Gradescope Practice



%\rhead works just like \lhead (you can also use \chead)

\rhead{\footnotesize Scott (mscott04)} 

\cfoot{\footnotesize \thepage} 





\begin{document}

\thispagestyle{empty} % This command disables the header on the first page. 



\begin{tabular}{p{15.5cm}} % This is a simple tabular environment to align your text nicely 

{\large \bf \sc{Mathematics Qualifying Exam:} Study Guide  } \\ %Course Number: Course Name

Tufts University \\ Fall 2021 - Spring 2023\\ Mitchell Scott (mscott04)  \\ %Professor's name

\hline

\\

\end{tabular} % Our tabular environment ends here.

\section{Numerical Analysis/ Linear Algebra}
\subsection{Systems of Equations}
\subsubsection{Linear Systems of Equations}
\paragraph{Gaussian elimination}
\paragraph{LU decompositions for full and sparse matrices}
\paragraph{Cholesky for full and sparse matrices}
\paragraph{Operation Counts}
\paragraph{Stability of Linear Systems (Condition Number)}
\paragraph{Stability of Gaussian elimination}
\paragraph{Basic Iterative Methods}
The basic goal of iterative methods is to solve $A\vec{x} = \vec{b}$, but by splitting of $A$:
\begin{align}
	A &= M - N\\
	A\vec{x} &= M\vec{x} - N\vec{x}\\
	M\vec{x} &= \vec{b} + N\vec{x}
\end{align}
But this leads to the questions, what is $M$? Well matrix $M$ is typically chosen so that the linear system $M\vec{z} = \vec{f}$ is ``easily solvable'' for any $\vec{f}$ either because it becomes a special type of matrix like $M$ might be diagonal, triangular, or triangular. It is also important to note that $M$ must not be singular, because then it isn't invertable, which is crucial in solving this problem. 

This means that we start with an initial guess and slowly refine it to get closer and closer to the actual answer. We set up the iterative system
\begin{align}
	Mx_{(k+1)} &= b + Nx_{(k)}
\end{align}
This means we can take the general formula $x_{k+1} = Gx_k + c$, where $G = M^{-1}N$ and $c = M^{-1}b$.
This has the Jacobian matrix that is 
\begin{align}
	G(x) = M^{-1}N
\end{align}
and to see if the iteration scheme is convergent if the spectral radius (recall that is the largest in magnitiude eigenvalue)
\begin{align}
	\rho(G) &= \rho(M^{-1}N)\\
	&< 1
\end{align}
The smaller $\rho(G)$ is, the faster we see convergence. 

You see though, we have a choice on what $\textbf{M}$ and $\textbf{N}$ are so that we get $\rho(\textbf{M}^{-1}\textbf{N})$ to be as small as possible. As with all things we care about, there is a trade off between the cost of solving the linear system with $\textbf{M}$ and how many iterations we care to take. The example is if $\textbf{M} = \textbf{A}$, then when we solve it, it can be done in one iteration, but that is a direct solve which can be very computational expensive.
\subparagraph{Jacobi}
Like I said before, we normally pick $M$ such that it has a special property to make it, nicely invertable and with a small norm. One way to do this is with the \underline{Jacobi Method}. This is where we have $A$
\begin{align}
	A &= D + L + U\\
	M &= D, \quad N = -(L + U)
\end{align}
This means that we get the iterative system
\begin{align}
	x^{(k+1)} &= \textbf{D}^{-1}\big( \vec{b} - (\textbf{L}+\textbf{U})x^{(k)}\big)
\end{align}
Now if we look at the overall solution components as opposed to the vectorized code, we get
\begin{align}
	x_i^{(k+1)} &= \frac{b_i - \sum_{j\neq i} a_{ij}x_j^{(k)}}{a_{ii}}, \quad i = 1,2,...,n
\end{align}
	\begin{itemize}
		\item \textbf{CON:} It is easy to see that we need to double store $x$ here, because we are using all of the old values to record the new values. Or, there is no update to $x$ so we need to let it go all the way through and then we can update it for the next iteration.
		\item \textbf{CON:} Its convergence rate is rather slow 
		\item \textbf{CON:} It is not guaranteed to converge, but
		\item \textbf{PRO:} It normally converges given standard practice. (e.g. if the matrix is diagonally dominant by rows)
	\end{itemize}
\subparagraph{Gauss - Seidel}
We can forgive a lot, but a slow convergence is not something that we want to forgive. It is so slow because it doesn't take advantage of the benefit of the new information available. However, \underline{Gauss-Seidel} does this by using the new component of the solution as soon as it is completed.
In terms of matrix calculations, we see that 
\begin{align}
	\textbf{M} &= \textbf{D} + \textbf{L}\\
	\textbf{N} &= -\textbf{U}\\
	x^{(k+1)} &= \textbf{D}^{-1}\big( \vec{b} - \textbf{L}x^{(k+1)} - \textbf{U}x^{(k)}\big)\\
	&= (\textbf{D} + \textbf{L})^{-1} \big( \vec{b} - \textbf{U}x^{(k)}\big)
\end{align}
Or if we care about the specific elemental components, we could get
\begin{align}
	x_i^{(k+1)} &= \frac{b_i - \sum_{j<i} a_{ij}x_j^{(k+1)} - \sum_{j>i} a_{ij}x_j^{(k)}}{a_{ii}}, \quad i = 1,..., n
\end{align}
\begin{itemize}
	\item \textbf{PRO:} Faster Convergence than Jacobi
	\item \textbf{PRO:} Don't have to keep two versions of $\vec{x}$, as we are constantly overwriting it.
	\item \textbf{CON:} Has to do these recursively, and we can't use this on a parallel computing system.
	\item \textbf{CON:} Doesn't always converge, but
	\item \textbf{PRO:} Converges under most practical applications, and less required than Jacobi (e.g. SPD matrix).
	\item \textbf{CON:} Still kind of slow to converge. 
\end{itemize}
\subparagraph{Successive Overrelaxation method}
\paragraph{Conjugate Gradient Method}

\subsubsection{Eigenvalue Problems}
\paragraph{Gerschgorin theorem}
\paragraph{power method} 
\paragraph{inverse power method} 
\paragraph{stability of eigenvalue problems}

\subsubsection{Nonlinear System of Equations, Optimization}
\paragraph{Newton's Method}
\paragraph{Quasi-Newton Methods}
\paragraph{Fixed Point Iteration}
\paragraph{Newton and Levenberg-Marquardt Methods for Unconstrained Optimization}

\subsection{Numerical Aproximation}
\subsubsection{Interpolation}
\paragraph{Interpolating polynomials}
\subparagraph{Lagrange Interpolating Polynomials}
\subparagraph{Hermite Interpolating Polynomials}
\paragraph{Runge phenomena}
\paragraph{Splines}
\paragraph{least squares approximation of functions and orthogonal polynomials}

\subsubsection{Integration}
\paragraph{Newton-Cotes methods}
\paragraph{Gaussian quadrature}
\paragraph{Euler-MacLaurin formula}
\paragraph{Adaptive quadrature}

\subsubsection{Differential Equations}
\paragraph{Convergence of explicit one-step methods}
\paragraph{Stiffness}
\paragraph{A- stability}
\paragraph{Impossibility of A-stable explicit Runge-Kutta methods}

\subsection{References}
\begin{enumerate}
	\item An Introduction to Numerical Analysis by K. E. Atkinson (Wiley)
	\item Unconstrained Optimization by P. E. Frandsen, K. Jonasson, H. B. Nielsen, and O. Tingleff
	\item Analysis of Numerical Methods by E. Isaacson and H. B. Keller (Wiley, Dover reprint)
	\item Finite Difference Methods for Ordinary and Partial Differential Equations by R. LeVeque (SIAM)
	\item A First Course in the Numerical Analysis of Differential Equations by A. Iserles (Cambridge University Press)
\end{enumerate}

\section{Analysis}

\section{Algebra}

\section{Partial Differential Equations}

\section{Algebraic Topology}

\section{Geometry}

\end{document}
